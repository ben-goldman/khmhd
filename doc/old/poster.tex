\documentclass{beamer}
\usepackage{graphicx}
\usetheme{posterben}
\usepackage{amsmath}

\title{Modeling Magnetic Field Amplification in Neutron Star Mergers}
\author{Ben Goldman}
\institute{Columbia University}

\usepackage{natbib}
\usepackage[scale=1.4]{beamerposter}
\geometry{papersize={41in,36in}}
\newcommand{\myfig}[4]{
  \begin{figure}
    \centering
    \includegraphics[width=#2\textwidth]{figures/#1}
    \def\param{#4}
    \ifx\param\empty
    \else
      \caption{\param}
    \fi
    \label{fig:#3}
  \end{figure}
}

\addtobeamertemplate{headline}{
  {\begin{tikzpicture}[remember picture,overlay] 
      \node [shift={(-2in,-2in)}] at (current page.north east) {\includegraphics[height=2in]{figures/THEA_logo}}; 
  \end{tikzpicture}}
  {\begin{tikzpicture}[remember picture,overlay] 
      \node [shift={(2in,-2in)}] at (current page.north west) {\includegraphics[height=2in]{figures/columbia.pdf}}; 
  \end{tikzpicture}}
}

\begin{document}

\begin{frame}[t]
  \begin{columns}
    \begin{column}{0.5\textwidth}
      \begin{block}{Question}
        {\bf How do colliding neutron stars transform their kenetic energy into magnetic energy?}
      \end{block}
      \begin{block}{Background}
        \begin{description}
          \item[Neutron Star (NS):] Extremely dense core of collapsed massive stars
          \item[Instability:] System in which perturbations grow exponentially
          \item[Turbulence:] Disorderly fluid state characterized by vortex motion and energy dissipation
          \item[Dynamo:] Self-sustaining process where turbulence within a conducting fluid amplifies an existing magnetic field
        \end{description}
        %\[\frac{\partial\vec{u}}{\partial t} + (\vec{u}\cdot \nabla)\vec{u} = -\frac{1}{\rho}\nabla p + \vec{g} + \nu\nabla^2\vec{u}\]
      \end{block}
      \begin{block}{Neutron star mergers}
        \begin{itemize}
          \item Extremely strong magnetic field produces gamma-ray burst and particle jets.
          \item NSes merge {\bf very} rapidly during coalescence.
        \end{itemize}
        \myfig{timeline.png}{0.7}{timeline}{}
      \end{block}
      \begin{block}{Past research}
        \begin{itemize}
          \item Applied GRMHD (general relativity magnetohydrodynamics) to simulate coalescence phase of neutron star collision \citep{palenzuela2022}.
          \item Magnetic field amplified in Kelvin-Helmholtz instability in core of merging system.
          \item Required to mathematically approximate dynamo, rather than simulate turbulent flow due to immense computational complexity of simulation.
        \end{itemize}
        \myfig{merger.png}{0.8}{merger}{Magnetic field development in simulated neutron star collision \citep{palenzuela2022}}
      \end{block}
    \end{column}
    \begin{column}{0.5\textwidth}
      % \myfig{cubes.pdf}{}{0.8}{cubes}
      \begin{block}{The Kelvin-Helmholtz instability}
        % https://ichef.bbci.co.uk/news/976/cpsprodpb/3BA9/production/_127937251_screenshot2022-12-08at2.03.52pm.png
        \begin{columns}[t]
          \column{0.6\textwidth}
          \begin{itemize}
            \item Opposing fluid velocity layer produces unstable system.
            \item Initially when perturbed, wavelike "lumps" develop.
            \item Eventually becomes turbulent as waves grow and interact.
          \end{itemize}
          \vspace{1in}
          \begin{flushright} Mature Kelvin Helmholtz instability in the Earth's atmosphere (Rachel Gordon/BBC)\end{flushright}
          \column{0.4\textwidth}
          \myfig{clouds.png}{1.0}{clouds}{}
        \end{columns}
      \end{block}
      \begin{block}{Methods}
        \begin{itemize}
          \item Used spectral MHD solver SpectralDNS \citep{mortensen2016} to simulate the Kelvin Helmholtz insatability on Columbia Ginsburg cluster.
          \item Initialized model with weak magnetic field and small velocity perturbations.
          \item Ran simulation until magnetic field stabilized.
          \item Recorded magnetic energy spectrum and growth rate.
        \end{itemize}
      \end{block}
      \begin{block}{Results}
        \myfig{images.pdf}{0.9}{images}{}
        \begin{columns}[t]
          \begin{column}{0.3\textwidth}
            \begin{itemize}
              \item 3-dimensional waves and eddies develop before turbulence onset.
              \item Magnetic field grows first during wavelike KH growth phase and then again during early development of turbulence.
              \item Kinetic energy dissipates during turbulence while magnetic field stays stable.
            \end{itemize}
          \end{column}
          \begin{column}{0.7\textwidth}
            \myfig{lines.pdf}{1.0}{lines}{}
          \end{column}
        \end{columns}
      \end{block}
        \begin{columns}[t]
        \begin{column}{0.5\textwidth}
      \begin{block}{Conclusions}
        \begin{itemize}
          \item The Kelvin Helmholtz instability can produce an efficient dynamo.
          \item Magnetic field is amplified mostly during early coalescence.
        \end{itemize}
      \end{block}
      \end{column}
          \begin{column}{0.5\textwidth}
            \begin{block}{Next steps}
        \begin{itemize}
          \item Rerun simulations under different choices of Reynolds number.
          \item Introduce angular momentum of neutron stars.
        \end{itemize}
      \end{block}
      \end{column}
    \end{columns}
    \end{column}
  \end{columns}
  \begin{block}{References}
    \nocite{burns2020a}
    \bibliographystyle{apalike}
    \bibliography{refs.bib}
  \end{block}
\end{frame}

\end{document}
